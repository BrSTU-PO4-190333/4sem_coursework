\newpage

\begingroup
  \section*{СПИСОК ИСПОЛЬЗОВАННЫХ ИСТОЧНИКОВ}
  \phantomsection
  \addcontentsline{toc}{section}{СПИСОК ИСПОЛЬЗОВАННЫХ ИСТОЧНИКОВ}

  \renewcommand{\addcontentsline}[3]{}% Remove functionality of \addcontentsline
  \renewcommand{\section}[2]{}% Remove functionality of \section

  \begin{thebibliography}{}
	\bibitem{youtube_start_learn_nn}
	Как я начал изучать нейросети и python - YouTube
	[Электронный ресурс].
	Режим доступа: \url{https://www.youtube.com/watch?v=PHdw0Uk_Lc4}.

	\bibitem{youtube_keras_7}
	Keras - установка и первое знакомство | \#7 нейросети на Python - YouTube
	[Электронный ресурс].
	Режим доступа: \url{https://www.youtube.com/watch?v=BQg9OZdzLLE}.

    \bibitem{youtube_start_learn_nn}
	Keras - обучение сети распознаванию рукописных цифр | \#8 нейросети на Python - YouTube
	[Электронный ресурс].\\
	Режим доступа: \url{https://www.youtube.com/watch?v=oCXh_GFMmOE}.

	\bibitem{moreDomains}
	Как нейронная сеть распознает цифры | \#9 нейросети на Python - YouTube
	[Электронный ресурс].
	Режим доступа: \url{https://www.youtube.com/watch?v=S3cViFMiYZ4}.
  \end{thebibliography}
\endgroup

\newpage
