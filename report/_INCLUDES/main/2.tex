\section{Анализ и описание используемого фреймворка}

\subsection{TensorFlow}

\subsubsection{История}
TensorFlow - открытая программная библиотека для машинного обучения,
разработанная компанией Google для решения задач построения и тренировки нейронной сети
с целью автоматического нахождения и классификации образов,
достигая качества человеческого восприятия.
Применяется как для исследований,
так и для разработки собственных продуктов Google.
Основной API для работы с библиотекой реализован для Python,
также существуют реализации для R, C Sharp, C++, Haskell, Java, Go и Swift.

Является продолжением закрытого проекта DistBelief.
Изначально TensorFlow была разработана командой Google Brain для внутреннего использования в Google,
в 2015 году система была переведена в свободный доступ с открытой лицензией Apache 2.0. 

\subsubsection{Лицензия}
TensorFlow доступна по лицензии Apache License 2.0.
Это дает нам возможность редактировать сам фреймворк.
В ходе курсового проекта нам не понадобится редактировать фреймворк.

\subsubsection{Операционная система}
TensorFlow доступен на разных операционных системах:
Windows, Android, iOS, Linux, MaсOS.
В ходе курсового проекта была выбрана операционная система Linux Debian 10 Xfce.

\subsubsection{Язык}
Чтобы писать на TensorFlow мы можем использовать выскоуровневые языки
такие как C++, Python, JavaScript.
В ходе курсового проекта был выбран язык Python.

\subsubsection{Версия}
Последняя версия TensorFlow имеет версию 2.5.0.

Узнать версию можем через \verb|tensorflow.__version__|.

\begin{lstlisting}[language=Python,]
    import tensorflow
    print(f'TensorFlow verion : {tensorflow.__version__}')
\end{lstlisting}

С версии 2.0 TensofFlow использует GPU.
То есть на Windows, Linux требуется ещё устанавливать CUDA for CPU TensorFlow.
TensorFlow можно запустить в двух режимах:
1) c поддержкой графического процессора,
2) без поддержки графического процессора.

Для такого, чтобы включить поддержку графического процессора,
то в Python используем библиотеку os.

\begin{lstlisting}[language=Python,]
    import os
    os.environ['TF_CPP_MIN_LOG_LEVEL'] = '2'
\end{lstlisting}

\subsection{Keras}

\subsubsection{История}
Keras - открытая нейросетевая библиотека,
написанная на языке Python.
Она представляет собой надстройку над фреймворками
Deeplearning4j, TensorFlow и Theano.
Нацелена на оперативную работу с сетями глубинного обучения,
при этом спроектирована так,
чтобы быть компактной,
модульной и расширяемой.
Она была создана как часть исследовательских усилий проекта ONEIROS
(англ. Open-ended Neuro-Electronic Intelligent Robot Operating System),
а ее основным автором и поддерживающим является Франсуа Шолле
(фр. François Chollet), инженер Google.

TensorFlow является фреймворком низкоуровневым.
Один разработчик Google написал свою надстройку Keras.
В итоге Google внедрила keras в сам tensorflow.
В ходе курсового проекта для удобства моделирования будет использоваться надстройка Keras.

\subsubsection{Лицензия}
Keras доступен по лицензии MIT. То есть мы имеем полное право её модифицировать. Но в ходе курсового проекта нам этого делать не прийдется.

\subsubsection{Операционная система}
Keras является мультиплатфоренный.
Так как мы используем Linux,
то библиотека для курсового проекта нам подходит.

\subsubsection{Язык}
Keras остается нашей библиотекой для фреймворка TensorFlow.
В ходе курсового проекта выбран высоко уровневый язык Python.
То есть Keras также будем использовать в Python.

\subsubsection{Версия}
Так как Keras мы используем как библиотеку включенную поумолчанию в TensorFlow,
то версия Keras совпадает с версией TensorFlow.

Узнать версию можем через \verb|tensorflow.keras.__verison__|.

\begin{lstlisting}[language=Python,]
    from tensorflow import keras
    print(f'Keras verion : {tensorflow.keras.__version__}')
\end{lstlisting}