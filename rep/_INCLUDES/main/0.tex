\section*{ВВЕДЕНИЕ} % Секция без номера
\addcontentsline{toc}{section}{ВВЕДЕНИЕ} % Добавить в содержание

Слово <<интеллект>> (от. лат. intellectus - ум, рассудок, разум) означает способность к мышлению. Ещё при развитии вычислительной техники была проблема создания систем интеллектуальной деястельности - искусственного интелекта. Это проблема поднималась из-за того, что многие задачи не могут быть решены точными алгоритмическими методами. Также эта сеть должна была не только уметь функциорировать в неловкой ситуации, но и уметь обучаться.

При обучении нейронный сети у нас будет нейкая выборка. Выборка - какое-то количество образцов (10\%, 50\%, 90\% или все 100\%).

Исскуственная нейронная сеть может выполнять разные задачи, например, прогнозирование, распознавание образов.

В задаче прогнозирования мы можем дать мы можем, например, дать 7 образцов градусов Цельсий: -40, 10, 0, 8, 15, 22, 38; и ожидаемый 7 образцов градусов Фаренгейт: -40, 14, 32, 46, 59, 72, 100; которые выводятся по формуле $F = C * 1.8 + 32$. Прогав эти образы, например, $50$ раз по сети мы настроим веса (найдем эти числа $1.8$ и $32$). Теперь дав новый образец в виде градуса Цельсий мы получим наш результат в градусах Фаренгейт, которые нейронная сеть хорошо определяет (подав $100$, получим $211.74736 \approx 212$), где $100 \cdot 1.8 + 32 = 212$. То есть научили сеть прогнозировать.

В задаче распознования мы можем разделять образы на классы. Например, в качестве образов мы можем использовать картинки кошек и собак в количестве 25 тысяч. Нейронная сеть прогнав даже 3 эпохи сможет распознавать на картинке кошку или собаку.

В ходе данного курсового проекта необходимо смоделировать нейронную сеть, которая сожеть распознавать 10 классов (10 цифр): нуль, один, два, три, четыре, пять, шесть, семь, восемь, девять.

Для задачи прогнозирования градусов Фаренгейт мы использовали простейшую нейронную сеть:

\begin{itemize}
    \item один входной нейрон (для значения градуса Цельсий)
    \item один выходной нейронн (для значения градуса Фаренгейт)
    \item связь между нейроном входным и выходным (вес со значением $1.8$)
    \item один нейрон <<биас>> (который всегда хранит значение 1)
    \item связь между нейронном <<биас>> и выходным (вес со значением $32$)
\end{itemize}

Для задачи распознавания образов мы будет использовать сложную математическую модель нейронной сети - рекурентную нейронную сеть, которая будет сканировать картинку не по одному пикселю, а например по 3 на 3, находя какие-то исгибы. С исгибов собирая уши, усы, глаза, на основе который нейронная сеть будет классифицировать.