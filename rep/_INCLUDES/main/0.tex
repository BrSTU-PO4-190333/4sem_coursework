\section*{ВВЕДЕНИЕ} % Секция без номера
\addcontentsline{toc}{section}{ВВЕДЕНИЕ} % Добавить в содержание

Слово <<интеллект>> (от. лат. intellectus - ум, рассудок, разум)
означает способность к мышлению.
Ещё при развитии вычислительной техники была проблема создания систем
интеллектуальной деястельности - искусственного интелекта.
Это проблема поднималась из-за того,
что многие задачи не могут быть решены точными алгоритмическими методами.
Также эта сеть должна была не только уметь функциорировать в неловкой ситуации,
но и уметь обучаться.

Исскуственная нейронная сеть может выполнять разные задачи,
например, прогнозирование, распознавание образов.

В задаче прогнозирования мы можем дать,
например, дать 7 образцов градусов Цельсий
и ожидаемый 7 образцов градусов Фаренгейт,
которые выводятся по формуле:

$$F = C \cdot 1.8 + 32$$

Прогав эти образы, например, 50 раз по сети,
мы настроим веса (найдем эти числа $1.8$ и $32$).
Теперь дав новый образец в виде градуса Цельсий,
мы получим наш результат в градусах Фаренгейт,
которые нейронная сеть хорошо определяет
(подав $100$, получим, например, $211.74736 \approx 212$).

$$100 \cdot 1.8 + 32 = 212$$

То есть научили сеть прогнозировать.

В задаче распознования мы можем разделять образы на классы.
Например, в качестве образов мы можем использовать картинки кошек и собак в количестве 25 тысяч.
Нейронная сеть прогнав даже 3 эпохи сможет распознавать на картинке кошку или собаку.

В ходе данного курсового проекта необходимо смоделировать нейронную сеть,
которая может распознавать не 2 класса,
а 10 классов (10 цифр):
нуль, один, два, три, четыре, пять, шесть, семь, восемь, девять.
